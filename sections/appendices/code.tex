%!TEX root=../../main.tex

This appendix summarises the source code deliverables of the project.

Starting on the second page, the log of all code commits over the source tree
of the next Invenio version is included, as recorded by the Git version control
system. The contributions can also be found online, at
\url{http://invenio-software.org/repo/personal/invenio-apanescu}. Note that the
\texttt{pu} (\textit{``Proposed Updates''}) branch refers to the next Invenio
version, while the \texttt{labs} branch is used for the Invenio Labs
demonstrator. The main work of the project, the annotation facilities, can be
found in the \texttt{anno} branch, with topical commits for each implemented
unit.

Similarly, at \url{http://github.com/adrianp/invenio-demosite} the contributions
brought to the separate package used for deploying customised Invenio
installations can be found. Of interest is the \texttt{labs} branch, used for
Invenio Labs.

Two external software libraries have been maintained in order to fit the
project's requirements. Namely, PyLD (\url{http://github.com/adrianp/pyld}),
which was patched in order to function with the Python 2.6 version used for
Invenio Labs (see \texttt{python2.6} branch), and Intro.js
(\url{http://github.com/adrianp/intro.js}), a JavaScript library used for the
Invenio Labs interactive tutorial feature (see \texttt{inveniolabs} branch).
A bug fix for Intro.js has been accepted by the library authors and merged
upstream (see \url{http://github.com/usablica/intro.js/pull/236}).

\clearpage

\lstinputlisting[showspaces=false,
                 basicstyle=\footnotesize,
                 breakatwhitespace=true,
                 breaklines=true,
                 morekeywords={SHA,Author,Date},
                 showstringspaces=false,
                 showtabs=false,
                 stepnumber=2,
                 numbersep=4pt]
  {static/lst/pu_commits.txt}

\clearpage
