%!TEX root=../../../main.tex

As JSON is a widely used format for transferring Web data, it seems only
natural that a method for specifying linked constructs using it to be proposed.
This is what JSON-LD achieves by allowing to encode RDF graphs in JSON
documents \cite{ref:jsonld}.

For this to be possible, the specification defines a number of keywords, all
preceded by the ``@'' token; the most commonly used ones are:
\begin{itemize}
  \item \texttt{@context}: used to define the keys and possibly values employed
                           inside the JSON document; for example, if JSON
                           is used to describe a person identified by a name,
                           the context could specify that the \textit{``name''}
                           field has the meaning as stipulated in a standard
                           ontology (e.g., Friend of a friend
                           (FOAF) \cite{ref:foaf}). The context is itself a JSON
                           object, where the keys are the various elements used
                           inside the main document and the values are
                           identifiers (e.g., IRI of an ontology).
  \item \texttt{@id}: used to uniquely identify JSON elements, usually by
                      means of an IRI; coming back to the above example, the
                      person could be uniquely identified by a homepage such as
                      \texttt{http://example.com/JohnDoe/}.
  \item \texttt{@type}: used to specify the data type of values (or more
                        generally of RDF nodes); for example, person names have
                        string type.
  \item \texttt{@value}: used to specify the actual data associated with a
                         property in the RDF graph. In the example above, the
                         \textit{``name''} property could have the value
                         \textit{``John Doe''}.
\end{itemize}

\newpage

A JSON-LD document example is included in Fig. \ref{lst:jsonld}. To transform
such a representation to a RDF graph, three simple operations need to be
applied\footnote{The complete algorithm is described formally in the
documentation.}:
\begin{enumerate}
  \item Expansion: this simply removes the context by replacing the keys inside
                   it with the values (usually IRI) across the main object.
  \item Flattening: coverts the JSON object to a list of RDF graph nodes.
                    Basically, it transforms the graph JSON structure into a
                    flat \textit{deterministic} one, in which all properties of
                    a node (identified by \texttt{@id}) are collected in a
                    single JSON object. The resulting graph is represented as
                    a list, the flat structure being easier to parse in an
                    automated manner. Figure \ref{lst:ldflat} provides the
                    flattened version of the document in Fig. \ref{lst:jsonld}.
  \item Conversion: from the flattened form to RDF triples. The result of this
                    final step on the document in Fig. \ref{lst:jsonld} is
                    included in Fig. \ref{lst:nquads}.
\end{enumerate}

\begin{figure}[!ht]
  \lstinputlisting[frame=tb,
                   captionpos=b,
                   numbers=none,
                   basicstyle=\ttfamily,
                   showspaces=false,
                   showstringspaces=false,
                   showtabs=false,
                   stepnumber=2,
                   morekeywords={@context,@id,@type,@value},
                   numbersep=4pt]
    {static/lst/ld.json}
    \caption[JSON-LD document describing a person]
            {JSON-LD document describing a person. The context defines the
             non-keyword elements used within the document, in order to provide
             information to the JSON-LD consumers regarding the semantic
             meaning of data.}
    \label{lst:jsonld}
\end{figure}

\begin{figure}[!ht]
  \lstinputlisting[frame=tb,
                   captionpos=b,
                   numbers=none,
                   basicstyle=\ttfamily,
                   showspaces=false,
                   showstringspaces=false,
                   showtabs=false,
                   stepnumber=2,
                   morekeywords={@context,@id,@type,@value,@graph},
                   numbersep=4pt]
    {static/lst/ldflat.json}
    \caption[Expanded and flattened JSON-LD document]
            {Expanded and subsequently flattened version of the JSON-LD document
             in Fig. \ref{lst:jsonld}. The terms which are not keywords in the
             JSON-LD vocabulary are replaced by their values from the original
             context, and the properties of each RDF graph node are collected in
             a single object. Therefore, the graph (\texttt{@graph}) will
             contain two nodes, one for each person.}
    \label{lst:ldflat}
\end{figure}

\begin{figure}[!ht]
  \lstinputlisting[frame=tb,
                   captionpos=b,
                   basicstyle=\ttfamily,
                   numbers=none,
                   showspaces=false,
                   showstringspaces=false,
                   showtabs=false,
                   stepnumber=2,
                   numbersep=4pt]
    {static/lst/nquads.xml}
    \caption[Information extracted from JSON-LD document in RDF N-Quads format]
            {Information extracted from the JSON-LD document in
             Fig. \ref{lst:jsonld} in RDF N-Quads format. The five statements,
             ended by full stop, specify that the resource identified by the IRI
             \texttt{http://example.com/JohnDoe/} is of type
             \textit{``person''}, with the meaning defined in the FOAF ontology,
             and has the name \textit{``John Doe''}, of type text (string) as
             defined in the Dublin Core Metadata Element Set (see
             \textit{``type''} term in \cite{ref:dc}). Moreover, this person has
             an acquaintance, \textit{``Jane Doe''}, described similarly.}
    \label{lst:nquads}
\end{figure}

Apart from being able to generate RDF graphs, this representation can be used
with another purpose, namely to obtain translated versions of the same data.
Consider a simple example in which a system supplies information about its
users in a format similar to the one in Fig. \ref{lst:jsonld}. Another
application wishes to use this information, but due to its internal workflow,
would need to obtain a JSON document in which the \textit{``name''} property
identifiers are replaced by \textit{``full-name''}.  Clearly, this could be
achieved by retrieving the JSON document from the first system and modifying
the fields, however, JSON-LD API implementations, such as PyLD \cite{ref:pyld},
provide another method, which basically can move the translation step from
consumer- to producer- side.  This method requires supplying another context
when retrieving the expanded (or flattened) JSON document, which specifies the
required translations; coming back to the above example, such a context can be:
\begin{verbatim}
  "@context": {"full-name": "http://xmlns.com/foaf/spec/#term_name"}
\end{verbatim}

Note that in both this context and the original one (Fig. \ref{lst:jsonld}),
the \textit{name} and \textit{full-name} keys, respectively, refer to the same
ontology IRI. Retrieving an expanded JSON-LD through this method, with a
context supplied, is a \textit{compaction} operation; the result will be a new
JSON document, in which the translations specified in the new context will be
applied (\textit{``name''} is replaced by \textit{``full-name''}). If no new
context is supplied, the expanded JSON-LD document is retrieved. This should be
the preferred communication method between applications interchanging JSON-LD
data. It is interesting to note that the specification allows sending an IRI to
a new context through an HTTP Link Header \cite{ref:rfc5988}, thus eliminating
the need for sending large JSON payloads.

While the above examples are trivial, more complex modelling and dissemination
operations can be achieved using JSON-LD. The next subsection gives an overview
on how this technology was employed for publicising annotations.
