%!TEX root=../../../main.tex

The implemented elements presented previously provide a strong basis upon which
future related projects can build upon, both in order to improve and extend the
existing annotation facilities.

In terms of the general framework, more of the currently existing Invenio
features for capturing user-side metadata (comments, reviews, tags, baskets)
can be adapted to using the new facilities in order to validate the design and
backend implementation details. Moreover, new facilities, such as file
attachments, decorated formatting, integration with Invenio's search engine,
full support for the access control system, or multimedia annotation
capabilities may be considered.

Regarding document targeted annotations, a more robust method of displaying the
concerned files might be of interest. For PDF documents, Mozilla's PDF.js
appears to be a viable solution, as long as a method of rendering complex files
on server-side is implemented, and certain cross-browser compatibility issues
are resolved; another option can be Multivio. Such a solution could allow for a
more attractive interface, employing which users could, for example, visually
highlight text fragments, in the same manner desktop applications allow.
Another direction regards the validation of user input; either a solution as
XTiger can be applied to restrict input on the user-side, or a more complex
grammar can replace the current regular expression backend. Moreover, a richer
collection of markers could be considered, covering, for example, the
differences between reviews targeting stylistic, grammatical, or factual
issues.

Finally, the REST  interface could be employed not only for data output, but
also for input; this may allow, for example, to import annotations from other
systems. The Invenio framework already supports data input through its REST
framework, as implemented, for example, by the ZENODO deposition API. Apart
from importing linked Web data, annotations from desktop applications such as
Mendeley can be of interest. On a higher level, exporting more Invenio resource
types, such as user accounts or record metadata, to JSON-LD compliant documents,
through the JSONAlchemy API, may provide a consolidated method of producing
linked data.
