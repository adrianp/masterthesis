%!TEX root=../../../main.tex

The CERN Document Server (CDS) is the main repository of the European
Organisation for Nuclear Research. Since 2002, it stores more than 1.300.000
records \cite{ref:cds} in over 500 collections \cite{ref:ludmilathesis}, both
related to the scientific activities of the organisation (research papers,
theses), and the administrative and public outreach activities. While the
majority of the records include textual content, images, videos and other data
formats are also distributed.

Apart from being the first user, CDS was also the initial motivation behind the
Invenio software platform. Due to the specific needs of the high-energy physics
community in terms of document publishing (e.g., collaborations comprising of
hundreds or even thousands of scientists\footnote{As of February 2012, the
ATLAS experiment included over 3000 scientist from 174 institutions
\cite{ref:atlas}.}) and dissemination (\cite{ref:annette} showed that users
prefer domain-specific search engines), a new solution that apart from
providing storage would also facilitate the entire workflow was deemed
necessary.

CDS also inspired the modular architecture of Invenio, as this facilitates its
deployment, administration and improvement: ``The key feature of CDS Invenio's
architecture lies in its modular logic. Each module embodies a specific,
defined, functionality of the digital library system. Modules interact with
other modules, the database and the interface layers. A module's logic,
operation and interoperability are extensible and customisable''
\cite{ref:lemeur}.
