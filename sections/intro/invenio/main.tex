%!TEX root=../../../main.tex

Invenio \cite{ref:invenio} is a software suite which allows running an online,
large-scale document repository or digital library. Its features cover all
aspects of such repositories, including document ingestion, classification,
ranking, indexing, curation and dissemination \cite{ref:kaplun, ref:glauner};
the architecture of the software is highly modular, each component being in
charge of one the mentioned aspects or other additional features. Currently, it
is being developed by an international developer community and is released as
open source software under the GNU General Public License version two.

Invenio complies with domain-specific standards such as the Open Archives
Initiative (OAI) metadata harvesting protocol. This is a low-barrier mechanism for
repository interoperability, in which HTTP requests can be made to providers in
order to request bibliographic record metadata \cite{ref:oai}.

The software platform is built using the Python programming language and
currently supports the MySQL relational database. Records are represented
internally using a Extensible Markup Language (XML) derivative of the MARC 21
bibliographic format, which is widely used for the representation and exchange
of bibliographic, authority, holdings, classification, and community
information data in machine-readable form \cite{ref:marc}.

The Invenio project originated in the world of high-energy physics, being used
in production by both the CERN\footnote{European Organisation for Nuclear
Research} Document Server (CDS) and the INSPIRE repository which is curated
by a collaboration of scientist from the Deutsches Elektronen-Synchrotron
(DESY), Fermi National Accelerator Laboratory (FNAL), and SLAC National
Accelerator Laboratory. Nevertheless, the continuous evolution of the project
made it suitable for other use-cases, such as the scientific information portal
of \'{E}cole Polytechnique F\'{e}d\'{e}rale de Lausanne, the Labordoc document
repository of the International Labour Organisation (ILO), or the European Open
Access implementations OpenAIRE and ZENODO; more information about repositories
using Invenio can be found at
\url{http://invenio-software.org/wiki/General/Demo}.
