%!TEX root=../../../main.tex

Invenio is a software suite which allows running an online document repository
or digital library. Its features cover all aspects of such repositories,
including document ingestion, classification, indexing, curation and
dissemination \cite{ref:invenio}; the architecture of the software is highly
modular, each component being in charge of one the mentioned aspects of other
additional features. It is released as Open Source software under the GNU
General Public License version two.

Invenio complies with domain-specific standards such as the Open Archives
Initiative (OAI) metadata harvesting protocol. This is a low-barrier mechanism for
repository interoperability, in which HTTP requests can be made to providers in
order to request bibliographic record metadata \cite{ref:oai}.

Invenio is built using the Python programming language and currently supports
the MySQL relational database. Records are represented internally using a XML
derivative of the MARC 21 bibliographic format, which is widely used for the
representation and exchange of bibliographic, authority, holdings,
classification, and community information data in machine-readable form
\cite{ref:marc}.

The Invenio project originated in the world of high-energy physics, being used
in production by both the CERN Document Server (CDS) and the INSPIRE
information system which is curated by a collaboration of scientist from the
Deutsches Elektronen-Synchrotron, Fermi National Accelerator Laboratory, and
SLAC National Accelerator Laboratory. Nevertheless, the continuous evolution of
the project made it suitable for other use-cases, such as the scientific
information portal of \'{E}cole Polytechnique F\'{e}d\'{e}rale de Lausanne, the
document repository of the International Labour Organisation, or the European
Open Access implementations OpenAIRE and ZENODO\footnote{More information about
repositories using Invenio can be found at
\url{invenio-software.org/wiki/General/Demo}}.
