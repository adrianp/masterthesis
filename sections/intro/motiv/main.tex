%!TEX root=../../../main.tex

The initial motivation for this project arose from a use-case originating on
CDS.  Here, the record commenting facilities of Invenio are often used by the
high-energy physics collaborations in order to review new papers in their
pre-publishing stage.  These collaborations use special workflows with multiple
commenting rounds for restricted drafts, necessary for amending potential
issues before making research results public \cite{ref:ludmila}.

In order to reference specific parts of the discussed papers, reviewers often
used a format similar to the following:
  \begin{itemize}
      \item \textit{``P1 - equation system is not presumed''}: a correction
          targeting the first page of the draft.
      \item \textit{``P1/L143 - equation system is not presumed''}: a variation
        of the above which also states the targeted text line.
      \item \textit{``Fig. 2 - equation system is not presumed''}: a comment on
        a figure.
      \item \textit{``P1, E2 - equation system is not presumed''}: a comment on
        Equation 2 on the first page.
  \end{itemize}
This format is also employed when the original paper authors respond to the
corrections, as shown in Fig. \ref{fig:comments}.

\begin{figure}[!ht]
  \centering
  \fbox{\includegraphics[scale=0.5]{static/img/comments.png}}
  \caption[Example of comments during the paper review phase on CDS]
          {Example of comments during the paper review phase on CDS. The author
           of a paper responds to corrections by referencing text line numbers.}
  \label{fig:comments}
\end{figure}

While being easy to use and clear regarding the corrections, this method also
presents a number of issues:
  \begin{enumerate}
    \item Non-standardised: even within the same commenting round of one draft,
      reviewers used different variations of the markup; for example, to
      reference the first page, both ``P1'' and ``Pag1'' could be used.
    \item Difficult to follow and aggregate: this issue emerged from the
      limitations of Invenio's commenting system which was not purposed for such
      structured input. Namely, both targeted remarks and free textual content
      could be combined in single comments, impeding reviewers from quickly
      identifying the required corrections to be made. Moreover, as commenting
      rounds can consist of hundreds of comments coming from multiple reviewers,
      aggregation is further hindered. For example, in a restricted CDS review
      round, we have identified over one thousand targeted remarks in 180
      comments coming from fifty persons.
  \end{enumerate}
Thus, a solution for solving these issues, while preserving the reviewing
workflow to which the end-users got accustomed to was required.

Aside from pre-publication reviews, such targeted notes could also be seen
useful for annotating resources (textual, multimedia or other formats). While
standard comments are sufficient for discussing general aspects regarding the
content, allowing annotations on various elements such as pages, figures or
paragraphs could prove useful for enriching the content and facilitating its
dissemination. Moreover, while comments often include a social aspect,
annotations could also be used in a private manner, for each user's self study.

Another motivation emerges from the need to consolidate a number of features
present in Invenio and its deployed variants (CDS, INSPIRE) which share a number
of similarities:
\begin{enumerate}
  \item Comments and reviews: allow general remarks on records; reviews include
    a ``\textit{star score}''.
  \item Tags: a new feature to be released in the next major version of Invenio,
    allows users to add brief annotations to records mainly in order to
    organise and categorise them. Can be public, personal to each user, or
    shared between groups.
  \item Baskets: allow users to create collections of records. Similar to tags,
    can be private or shared between users.
\end{enumerate}

Apart from these general Invenio features, certain deployments integrate their
own homologous features. One example is INSPIRE, which implemented a custom form
for allowing users to suggest corrections for the citation list of papers as the
one in Fig. \ref{fig:inspire}. This can be seen as a particular use-case of the
record review practices previously described, in which only the reference table
is considered.

\begin{figure}[!ht]
  \centering
  \fbox{\includegraphics[scale=0.5]{static/img/inspire.png}}
  \caption[Reference correction form as implemented by INSPIRE.]
          {Reference correction form as implemented by INSPIRE. Users are
           invited to suggest amends, a structured form being used for guiding
           data input.}
  \label{fig:inspire}
\end{figure}

A final issue motivating this work is related to the dissemination of metadata.
As mentioned, Invenio already includes features facilitating record hasvesting
in compliance with the OAI protocol, but currently does not implement any
specific mechanism for exporting the user-generated metadata.
