%!TEX root=../../../main.tex

The Invenio platform provides certain features, such as a commenting facility
or document collections, that allow users to discuss and augment bibliographic
records. While these cover a number of requirements, they do not satisfy all
the users' collaborative needs, especially those of large scientific
communities such as the ones using CDS. The project identified three possible
axis of improvement inside the Invenio ecosystem:
\begin{itemize}
    \item Targeted commenting: lower granularity levels should be allowed when
      annotating  data items. For example, document comments
      should support references to specific sections, paragraphs, or text lines.
    \item Rich feature set: various methods of augmenting data items, such as,
      but not limited to, document reviews, tags, or record collections should
      be delivered in a consolidated manner, having a common base upon which new
      similar features can build by extending existing components.
    \item Dissemination: communication with external entities should be possible,
      not only in terms of record bibliographic data, but also with regard to
      user-generated metadata. This includes both making internal data available
      to third-parties, and also the ingestion of items from other online or
      offline systems.
\end{itemize}

The project's motivation is further explained by the use-cases described in the
next subsection. Both a conceptual design and concrete implementation of a
solution to the three points above has been proposed, taking into account state
of the art technologies and patterns.
