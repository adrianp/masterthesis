%!TEX root=../../main.tex

Considering the current features of Invenio and the state of art, we
defined the following requirements for adding annotation capabilities to
the platform:
\begin{itemize}
  \item The system should be extensible and allow consolidating the existing
        commenting, reviewing, tagging and grouping of records either by
        superseding the existing feature set, or by providing a base framework
        upon which other features can build. Future similar capabilities should
        be considered by means of development time reduction.
  \item The integration of the new system and features should be backward
        compatible and transparent to end-users. Basically, the Invenio
        application should continue to function in the same manner, with an
        identical feature set, as before this project.
  \item The targeted commenting use-case, presented in Section \ref{sec:motiv}
        should be considered of high-priority, as the described workflow, while
        frequently used by the community, is inefficient.
  \item The new system should seamlessly integrate with the access control
        features of Invenio. For example, during the review period, records on
        CDS are restricted to only the participating authors and reviewers.
        Thus, any associated metadata should also obey the access control rules.
  \item Disseminating annotations should be possible in a standardised manner,
        alike the harvesting mechanisms employed for records.
\end{itemize}

The first considered option was using one of the systems previously presented,
namely Annotator or Pundit. The first one seems a promising solution, as it
uses a loose data model that can be easily extended to various use cases and
annotation target types. Unfortunately, the software is involved more on the
frontend aspect, with only minimal infrastructure for dissemination and
storage. Also, the project focuses more on annotating fragments of HTML
documents, which does not fit the targeted commenting use case; records in
Invenio provide the full text usually as PDF documents and such, a proper
solution must be implemented in order to allow making references inside them.

The Pundit project is more complete in terms of features, allowing to annotate
various content types out of the box (text, multimedia), and providing a rich
editor which allows embedding RDF graphs and attaching external files. This
option was designed more as a standalone application and not as a lightweight
wrapper like Annotator, providing features such as authentication or a
complete backend including both storage and dissemination by means of a REST
interface.  This made it unsuitable for a seamless integration with Invenio,
which would have meant stripping any unnecessary components while preserving
the required functionality.

One downside shared by both Annotator and Pundit is related to the targeted
document annotation exemplified in Fig. \ref{fig:comments}. While it is
possible to simulate such a feature in both systems, it is worth noting that
paper reviewers frequently take notes in offline mode (e.g., during commute),
this being one of the reasons behind the employed markup. Using any of the two
systems in offline mode for this purpose is mostly impossible, as they rely on
rich graphical interfaces for specifying the exact annotation targets. For
this project's purposes, both an interface and an offline editing mode should
be provided.

The careful evaluation of the existing alternatives lead us to consider
implementing a custom solution which, would satisfy all the user requirements,
and also allow for a gradual integration in the existing Invenio framework.
Nevertheless, existing components and ideas would be reused and standard
practices would be strictly followed, especially in terms of metadata
dissemination.

The next subsection will give an abstract overview of the proposed system,
while Section \ref{sec:impl} will detail the delivered feature set and
implementation details. Due to the overlap of this project with a major
refactoring of the Invenio project, certain miscellaneous tasks were carried
out in order to provide a working base for the new features, this being briefly
discussed in Subsection \ref{sec:v2}.
