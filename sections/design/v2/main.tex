%!TEX root=../../../main.tex

As the Invenio software is now being used at large-scale by a considerable
number of organisations across various activity domains, the ability of easily
adapting to various use cases became a stringent requirement. As a result, a
major refactoring of the project, in order to make it more customisable, has
been undertaken. The main focus was on changing the delivery package from a
fairly tightly integrated digital library platform to a coherent collection of
repository components. As this project overlapped with the reorganisation and
contributed a number of required additions and fixes to the main source tree, a
brief description of the process is included in the following.

The first step in rewriting the platform targeted the reorganisation of the
source code. Invenio already featured a modular design \cite{ref:lemeur}, but
further efforts were made to better separate utilities (record commenting, user
messaging etc.) from core bibliographic modules. This also allowed the
identification of components that could be reused outside Invenio, being
general enough for any other Web application. For example, a component offering
support for external user authentication sources is now distributed separately
of core Invenio.

The second important change regards the use of external tools for simplifying
the code base and workflows. A modern framework for building Web applications,
Flask \cite{ref:flask} was employed for replacing the custom Web Server Gateway
Interface (WSGI) components used since Invenio's first iterations. Similarly, a
Structured Query Language (SQL) toolkit and object--relational mapper (ORM),
SQLAlchemy \cite{ref:sqlalchemy}, was employed in order to simplify the code
used for interacting with the database backend and eliminate the need for hand
crafted queries. This is implemented as a thin wrapper around standard Python
classes, that abstracts database operations from developers. An added benefit
of using such a solution is the possibility of switching the database system;
the Invenio collaboration currently explores the move from MySQL to PostgreSQL
for its relational needs.

Other important backend changes involved the data workflow, both inside and
outside Invenio. Support for other formats than MARC was added by a new module,
JSONAlchemy, which is also of interest as it provides abstractions for handling
JSON documents in the same manner SQLAlchemy facilitates the handling of
relational models. Another module that facilitates the creation of REST
interfaces was deployed, a primer usage scenario being an Application
Programmable Interface (API) for document deposition.  Other additions include a
customisable record ingestion workflow engine, developed within the INSPIRE
project, and support for cloud storage by means of a filesystem abstraction
module.

Another targeted aspect relates to the user interface, a modern and
customisable solution being of high interest. Customisability has been achieved
by deploying a templating engine, Jinja version 2, which is already integrated
with the Flask framework, for creating highly modular HTML pages. It features
an extension model in which templates inherit blocks of markup from each other,
similar to object oriented inheritance; thus, if an organisation using Invenio
desires to use its own custom design elements, it just needs to extend or
overload the base templates, modifying only the required parts (e.g., logo or
colour scheme).

Modernisation wise, a series of HTML5 technologies have been employed; the
Bootstrap \cite{ref:bootstrap} frontend framework is being used for defining a
number of interface elements (dialogs, menus, etc.), and updated versions of
various JavaScript libraries for delivering a fluid user experience in line
with modern practices and patterns.

A new deployment process has been developed, replacing the GNU build system
with a Python-specific process for supplying the backend dependencies, and a
specialised workflow for static files, such as JavaScript libraries and style
sheets.

Performance wise improvements were also targeted, indirectly by adopting new
practices while re-witting legacy code, and directly by employing new
technologies such as Redis \cite{ref:redis} for distributed caching, or Celery
\cite{ref:celery} for task scheduling and queuing.

The contributions made by this project to the refactoring are as follows:
\begin{itemize}
  \item Fixed various issues regarding user account handling, such as
        authentication or validation. Also, certain features such as the
        password change facility were ported from the legacy code to the new
        structure.
  \item Contributed with fixes and additions to the developer workflow; for
        example, a brief summary of the steps required for setting up an Invenio
        development environment was compiled and maintained. Certain
        maintenance tasks were also carried out regarding the external
        dependencies used by Invenio, such as version upgrading and source
        discovery and correction.
  \item Improved various user interface aspects, by fixing existing issues,
        further modularising Jinja templates, reorganising source code and
        developing new utilities. For example, a routine for simplifying the
        internationalisation of text strings used inside JavaScript sources
        was developed.
  \item Reorganised the file tree structure for a number of modules, especially
        in regards to better separating JavaScript and Cascading Style Sheets
        (CSS).
  \item Tested the system in order to discover or confirm any issues resulting
        from the code reorganisation, and, where possible, proposed fixes.
\end{itemize}

Apart from these prompt contributions, a number of systematic and more
complex tasks were carried out. As, in the initial phase, the project built
around the commenting facilities already provided by Invenio, this module was
constantly maintained and improved, any general additions being pushed
back to mainline.

Other additions have been the result of identifying new requirements while
developing the annotation capabilities, namely:
\begin{enumerate}
  \item Implementing an easy-to-use annotation interface requires displaying a
        preview of the targeted documents, in order to provide the offline
        editor experience described in Section \ref{sec:comments}. While this
        is still an open problem in Invenio, a prototype of such a facility for
        PDF documents has been implemented in the existing previewer module,
        which was also improved to fit the new use-case. A brief research of
        existing solutions was carried out, PDF.js \cite{ref:pdfjs} and Multivio
        \cite{ref:multivio} being considered for future work.
  \item Extending the JSONAlchemy module in order to allow greater
        flexibility in terms of usage scenarios and allow exporting data in
        the JSON-LD RDF serialisation format. This will be discussed further in
        Subsection \ref{sec:diss}.
\end{enumerate}

Finally, in order to provide a demonstrator for the new annotation features, a
special instance of Invenio, Invenio Labs, was deployed at
\url{http://labs.invenio-software.org}. This required maintaining a running
version in an environment close to production ones, while also providing
customisation additions. A significant encountered issue relates to the Python
environment version (2.6) used on CERN servers running the Scientific Linux
operating system. For example, the external library used for generating JSON-LD
documents, PyLD \cite{ref:pyld} was back-ported in order to work properly.

Building this demonstrator was facilitated by the introduction of a new
distinct package, \texttt{invenio-demosite}, which allows developers to quickly
set up a custom Invenio instance. A number of modified HTML templates were
created, a website tutorial feature developed in JavaScript was developed, and
record data along with the necessary deployment routine were put in
place over this standardised customisation package.

The contributions made by this project in terms of source code are
enumerated in Appendix \ref{apx:code}, along with references to the involved
repositories.
