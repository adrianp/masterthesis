%!TEX root=../../main.tex

The aim of this project was to develop a structured annotation facility, the
original use-case being identified in the scientific paper pre-publication
workflow employed by high energy physics collaborations. Apart from being able
to comment at the global document level, collaboration members also require
more granular options, such as section, paragraph, figure, or equation targeted
annotations.  Moreover, aggregating such reviews should be possible in a facile
manner, which would simplify the workflow for the authors in charge of amending
papers.

After studying a number of use cases and state of the art solutions and related
technologies, a general annotation framework for the Invenio digital library
platform was designed and implemented. The aim was to provide a standardised
method of allowing input of end-user generated metadata regardless of its form:
document comments, record collections and tags, or Web page annotations.

The targeted annotator facility was implemented along with a document
previewer which facilitates both data input and aggregation for end-users.
Furthermore, a method of disseminating annotations, in line with the emerging
Open Annotation RDF data model, was implemented. It makes use of modern Web
technologies such as the JSON-LD serialisation format and REST services, in
order to bestow a streamlined circulation workflow for any third-party
interested in accessing user-generated metadata.

The annotation facilities were developed in the context of the new Invenio
version, to which the project brought a number of additional contributions.
These include fixes and additions to several components of the digital library
platform.

The document targeted annotator was released for end-user testing in a
controlled environment, the feedback being highly positive. The general
consensus is that the new aggregation features significantly simplify the
pre-publication review process; future improvements, such as the ones presented
in Section \ref{sec:future}, might improve the workflow even further.
