%!TEX root=../../main.tex

The project delivered a prototype of an annotation framework for the Invenio
digital library platform. The high-level aim for this was to provide a
standardised method of allowing the input of end-user generated metadata,
regardless of its form: document comments or tags, record collections, or
IRI-identified resource annotations. The more specific considered use-case, the
initial motivation of the project, was the one related to CDS paper review
workflow, in which authors need to aggregate large number of targeted comments
on specific document locations, such as pages, sections, figures or equations and
consider the necessary corrections.

The base building blocks of the framework are provided, along with
implementations of the Web page annotator and document targeted commenting
facility. Moreover, a method for disseminating annotation data, conforming with
Open Annotation RDF model, is included; this makes use of modern technologies
such as the JSON-LD serialisation format or REST Web services in order to
bestow a streamlined circulation workflow for any third-party interested in
accessing Invenio metadata.

A number of possible future improvements remain. In terms of the general
annotation framework, more of the currently existing Invenio features for
capturing user-side metadata (comments, reviews, tags, baskets) would need to
be adapted to using the new facilities in order to validate the design and
backend implementation details. Moreover, new facilities, such as file
attachments, decorated formatting, integration with Invenio's search engine,
full support for the access control system, or multimedia annotation
capabilities can be considered.

Regarding document targeted annotations, a more robust method of displaying the
concerned files is required. For PDF documents, Mozilla's PDF.js
appears to be a viable solution, as long as a method of rendering complex files
on server-side is implemented, and certain cross-browser compatibility issues
are resolved; another option can be Multivio. Such a solution could allow for
a more attractive interface, using which users could, for example, visually
highlight text fragments, in the same manner desktop applications allow.
Another direction regards the validation of user input; either a solution as
XTiger can be employed to restrict input on the user-side, or a more complex
grammar can replace the current regular expression backend. Moreover, a
richer collection of markers could be considered, covering, for example, the
differences between reviews targeting stylistic, grammatical, or factual issues.

Finally, the REST  interface could be employed not only for data output, but
also for input; this may allow, for example, to import annotations from other
systems. The Invenio framework already supports data input through its REST
framework, as implemented, for example, by the ZENODO deposition API. Apart
from importing linked Web  data, annotations from desktop applications such as
Mendeley can be of interest. On a higher level, exporting more Invenio resource
types, such as user accounts or record metadata, to JSON-LD compliant documents,
through the JSONAlchemy API, may provide a consolidated method of producing
linked data.

To conclude, the project explored a number of Web technologies linked to the
resource annotation use case, implemented solutions for certain requirements of
the Invenio platform, and paved the way for related future directions.
