%!TEX root=../../main.tex

The project delivered a prototype of an annotation framework for the Invenio
digital library platform. The high-level aim for this was to provide a
standardised method of allowing the input of end-user generated metadata,
regardless of its form: document comments or tags, record collections, or
IRI-identified resource annotations. The more specific considered use-case, the
initial motivation of the project, was the one related to CDS paper review
workflow, in which authors need to aggregate large number of targeted comments
on specific document locations, such as pages, sections, figures or equations
and consider the necessary corrections.

The base building blocks of the framework are provided, along with
implementations of the Web page annotator and document targeted commenting
facility. Moreover, a method for disseminating annotation data, conforming with
the Open Annotation RDF model, is included; this makes use of modern
technologies such as the JSON-LD serialisation format, or REST Web services in
order to bestow a streamlined circulation workflow for any third-party
interested in accessing Invenio metadata.

The project built upon the new version of Invenio digital library platform, to
which it brought a number of additional contributions, and the delivered
prototype was deployed in order to gather end-user reactions. The feedback was
positive, as the document annotator and its aggregation facilities can
significantly simplify the pre-publication review workflows of the scientific
community. Future developments might further improve this process, having the
existing elements as the foundation.

To conclude, the project explored a number of Web technologies linked to the
resource annotation use case, implemented solutions for certain requirements of
the Invenio platform, and paved the way for related future directions.
